\documentclass[pdf,hyperref={unicode}, aspectratio=43, serif,11pt]{beamer}
\usepackage[T2A]{fontenc}
\usepackage{graphicx}
\usepackage[english, russian]{babel}  
\DeclareGraphicsExtensions{.pdf,.png,.jpg}

%Задаем параметры документа
% \usepackage[top = 20 mm, 
%             bottom = 20 mm, 
%             left = 30 mm, 
%             right = 30 mm]{geometry}
            
%Красная строка в первом абзаце
\usepackage{indentfirst}

%Величина отступа красной строки
\setlength{\parindent}{12.5 mm}

%Межстрочный интервал
%\def\baselinestretch{1.5}
\usepackage{setspace}
\setstretch{1}

\title[Презентация 11ПИФМ]{История Физико-математического факультета}
\author{К.А. Алёшин}
\date{9 мая 2022}
\institute[]{Орловский государственный
университет имени И.\,С.~Тургенева}
\def\baselinestretch{1}

\usefonttheme[onlymath]{serif}
\usepackage{beamerthemesplit}

%тема оформления
\usetheme{Madrid}%Warsaw

%цветовая гамма
\usecolortheme{seahorse}%whale


\begin{document}
\begin{frame}
\titlepage
\end{frame}


\begin{frame}
	\frametitle{Зарождение}
	Пятого августа 1931 года СНК РСФСР распорядился создать в Орле индустриально-педагогический институт. 16 октября первые студенты, 121 человек, и 11 преподавателей приступили к занятиям на четырех факультетах: физико-техническом, химическом, социально-экономическом и политехническом.\\
	 В 1932 году индустриально-педагогический институт преобразован в педагогический, физико-технический факультет – в физико-математический. В этом же году открыты рабфак и вечернее отделение с двумя факультетами: физико-математическим и литературным. Открыто заочное отделение с теми же факультетами, что и на стационаре.

\end{frame}
\begin{frame}
	\begin{center}
	\includegraphics{1.jpg}
	\end{center}
Окончание математического и физического отделений дает право на работу в качестве преподавателя полной и неполной средней школы, рабфака и техникума, по соответствующей специальности – математики и физики.
\end{frame}
\begin{frame}
	\frametitle{Первый выпуск}
В июне 1935 года институт осуществил первый выпуск студентов. 45 специалистов (26 из них – математики) разъехались в разные районы страны. Диплом №1 был выдан студентке физмата Аносовой Антонине Николаевне.\\
Успеваемость студентов физико-математического факультета на первое сентября 1935 года составляла 87,7\%, на первое сентября 1936 года – 93,5\%. На «отлично» экзамены сдали 7,3\% студентов. Стопроцентная успеваемость была на третьем курсе математиков. Он был награжден переходящим Красным Знаменем института. Студентка этого курса З.Я. Власова как отличница была направлена в Москву на Всероссийское совещание отличников учебы студентов вузов.
\end{frame}

\begin{frame}
	\frametitle{Первая стенгазета}
	Студенческая жизнь никогда не ограничивалась только учёбой. В 1946 году на физмате создали стенгазету «Студент». Её редактором стал фронтовик, поступивший на факультет в 1949 году, Владимир Кречетов (в 1960-1970 гг. заместитель декана физмата). По его мнению, «главным достоинством «Студента» была конкретность. Не боялись называть фамилии, не боялись друга назвать тунеядцем, если он не готовился к занятиям. Все были верны одной цели – учиться, и поэтому критика принималась как должное. Осознание цели – учеба, – вот что двигало ребятами.
\end{frame}
\begin{frame}
	% Please add the following required packages to your document preamble:
	% \usepackage[table,xcdraw]{xcolor}
	% If you use beamer only pass "xcolor=table" option, i.e. \documentclass[xcolor=table]{beamer}
	\begin{table}
		\begin{tabular}{|c|c|}
			\hline

			{\color[HTML]{000000} Цифры направления} & Направление                         \\ \hline
			01.03.01, 01.04.01                       & Математика                          \\ \hline
			01.03.02, 01.04.02                       & Прикладная математика и информатика \\ \hline
			03.03.02, 03.04.02                       & Физика                              \\ \hline
			44.03.05, 44.04.01                       & Педагогическое образование          \\ \hline
			09.03.03, 09.04.03                       & Прикладная информатика              \\ \hline
		\end{tabular}
	\end{table}
	\begin{center}
		Направления, изучаемые на факультете физмат, в 2020
	\end{center}
\end{frame}

\end{document}